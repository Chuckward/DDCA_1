\documentclass[12pt,a4paper,titlepage,oneside]{article}

\usepackage{dideProtocol}
\sloppy


\exercise{Lab Exercise I}

% enter your data here
\authors{
  Andreas Ciachi, Matr. Nr. 1029176 \par
  {\small e1029176@student.tuwien.ac.at} \par
}

\begin{document}

\maketitle

\subsection*{Structural modeling}
\begin{figure}[ht!]
  \centering
  \framebox[\linewidth]{
    \includegraphics[scale=0.4, angle=90]{pictures/RTL_view_1} 
  }
  % \includegraphics[width=1.0\linewidth]{your filename here}
  \caption{Screenshot showing the top level design in the RTL netlist viewer.}
\end{figure}

\begin{figure}[ht!]
  \centering
  \framebox[\linewidth]{
    \rotatebox{30}{Insert your screenshot here.} 
  }
  % \includegraphics[width=1.0\linewidth]{your filename here}
  \caption{Screenshot of pin assignments.}
\end{figure}

\subsection*{Simulation}

\begin{figure}[ht!]
  \centering
  \framebox[\linewidth]{
    \rotatebox{30}{Insert your screenshot here.} 
  }
  % \includegraphics[width=1.0\linewidth]{your filename here}
  \caption{Three characters propagating from input to the displays}
\end{figure}

\begin{table}[ht!]
  \centering
  \caption{Timing measurements}
  \begin{tabular}{|l|l|}
    \hline
    Time                                                                 & Value  \\ \hline 
    First transition of character input on PS2 to ASCII character output & ? ns   \\
    ASCII character output to seven segment display update               & ? ns   \\
    ASCII character output to textmode instruction                       & ? ns   \\
    1/Display frame rate (\emph{vsync\_n} period)                        & ? ns   \\ \hline
  \end{tabular}
\end{table}

\begin{qa}
  \question{Different propagation delays: How long is the transition time you
    measured when the seven segment display output bus changes its value and
    multiple signals toggle?}
  \answer{...}
\end{qa}

\begin{qa}
  \question{Describe how the bug in the \emph{ps2\_ascii} component affects the design.}
  \answer{...}
\end{qa}


\begin{figure}[ht!]
  \centering
  \framebox[\linewidth]{
    \rotatebox{30}{Insert your screenshot here.} 
  }
  % \includegraphics[width=1.0\linewidth]{your filename here}
  \caption{Different propagation delays on the seven segment display bus.}
\end{figure}


\subsection*{Behavioral Modeling}

\begin{figure}[ht!]
  \centering
  \framebox[\linewidth]{
    \rotatebox{30}{Insert your screenshot here.} 
  }
  %%%%%%%%%%%%%%%%%%%%%%%% 
  % Include markers for measuring the duration of the frame (including the start-bit).
  %%%%%%%%%%%%%%%%%%%%%%%% 
  
  % \includegraphics[width=1.0\linewidth]{your filename here}
  \caption{Screenshot of a simulation showing the reception of a whole UART frame.}
\end{figure}

\begin{qa}
  \question{Which baudrate did you use for the above simulation?
    How long should the transmission take for the whole frame (including start and
    stop bit)?
    What is the time you measured in the simulation (not including the stop
    bit)?}
  \answer{...}
\end{qa}

\begin{table}[ht!]
  \centering
  \caption{Resouce usage of the serial module (including all submodules).}
  \begin{tabular}{|l|l|l|l|}
    \hline
    & LC Combinationals & LC Registers & Memory      \\ \hline 
    Absolute number      &                   &              &             \\
    \% of whole design   &                   &              &             \\
    \% of whole FPGA resources &                   &              &             \\ \hline
  \end{tabular}
\end{table}

\subsection*{Measurement}

\begin{figure}[ht!]
  \centering
  \framebox[\linewidth]{
    \rotatebox{30}{Insert your screenshot here.} 
  }
  %%%%%%%%%%%%%%%%%%%%%%%% 
  % Include markers for measuring the duration of the frame (including the
  % start-bit). Transmit an appropriate character such that you can perform this
  % measurement.
  %%%%%%%%%%%%%%%%%%%%%%%% 
  
  % \includegraphics[width=1.0\linewidth]{your filename here}
  \caption{Screenshot of a measurement showing the duration of a whole UART
    frame sent from the FPGA to the computer.}
\end{figure}

\begin{qa}
  \question{Which baudrate did you use for the hardware implementation?
    How long should the transmission take for the whole frame (including start and
    stop bit)?
    What is the time you measured (not including the stop
    bit)?}
  \answer{...}
\end{qa}

\begin{figure}[ht!]
  \centering
  \framebox[\linewidth]{
    \rotatebox{30}{Insert your screenshot here.} 
  }
  %%%%%%%%%%%%%%%%%%%%%%%% 
  % Include the hsync, vsync, r, g, b signals and markers for measuring the
  % duration of the seventh visible pixel row (hsync to hsync). Send characters
  % to the display before making the measurement so that the line is not all
  % black.
  %%%%%%%%%%%%%%%%%%%%%%%% 
  
  % \includegraphics[width=1.0\linewidth]{your filename here}
  \caption{Screenshot of a display timing measurement showing the seventh visible pixel row.}
\end{figure}

\begin{figure}[ht!]
  \centering
  \framebox[\linewidth]{
    \rotatebox{30}{Insert your screenshot here.} 
  }
  
  % \includegraphics[width=1.0\linewidth]{your filename here}
  \caption{Screenshot of the trigger setting for finding the seventh row.}
\end{figure}

\begin{qa}
  \question{How long is the hsync-to-hsync interval you measured?}
  \answer{...}
\end{qa}

\begin{qa}
  \question{What is the sampling rate you used for this measurement?}
  \answer{...}
\end{qa}

\begin{qa}
  \question{What is the uncertainty you have to add to the measured time when
    sampling with this sample rate?
    ($\pm$ ? ns)} \answer{...}
\end{qa}

\begin{qa}
  \question{What is the maximum frequency that you could reliably display as a
    switching waveform at this sample rate (assuming a 50\% duty cycle)?}
  \answer{...}
\end{qa}


\clearpage
\section*{Feedback \& Comments}
By answering the optional questions below you can give us feedback and help us 
to further  improve this lab course. Your answers will not influence 
your grading! 

\begin{qa}
  \question{How many hours did you need to solve this lab exercise?
    Please give us a rough estimate.}
  \answer{}  
\end{qa}

\begin{qa}
  \question{Were there any annoying problems you encountered (e.g. bugs in tools,
    flaws in the task description or documentation, etc.)?}
  \answer{}  
\end{qa}

\begin{qa}
  \question{Other remarks?}
  \answer{}  
\end{qa}


\end{document}
